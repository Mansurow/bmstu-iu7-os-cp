\chapter*{ВВЕДЕНИЕ}
\addcontentsline{toc}{chapter}{ВВЕДЕНИЕ}

В настоящее время UNIX-подобными операционными системами явялются одними из самых популярных в мире. На персональных компьютерных за последние 10 лет такие системы  встречаются у 20\% пользователей~\cite{statistics:using-desktop}, а на мобильных устройствах около 70\%~\cite{statistics:using-mobile}. Для пользователей данных систем важно предоставлять информацию о выполнении процесса, который постоянно используют память, посылают и принимаю сигналы, семафоры, сегменты разделяемой памяти и программные каналы. Особое внимание уделяется операционным системам Lunux, так как ядро Linux возможно изучить, благодаря тому что оно имеет открытый исходный код.

%Данная работа посвящена исследованию структр ядра, хранящих информацию о процессах, сигналов, семафорах, сегметов разделямой памяти и программных каналов.

Целью курсовой работы является разработка загружаемого модуля ядра, предоставляющего информацию о о приоритете, времени выполнения и простоя, выделенной виртуальной памяти, сегментах разделяемой памяти, программных каналов, семафорах и сигналов процессов.

Чтобы достигнуть поставленной цели, требуется решить следующие задачи:
\begin{itemize}
	\item провести анализ структур и функций, предоставляющих возможность реализовать поставленную задачу;
	\item разработать алгоритмы и структуру загружаемого модуля ядра, обеспечивающего отслеживание процессов.  
\end{itemize}
