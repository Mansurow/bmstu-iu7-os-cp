\chapter{Технологическая часть}

\section{Выбор языка и среды программирования}

Для реализации ПО был выбран язык программирования C~\cite{c}, поскольку в нём есть все инструменты для реализации загружаемого модуля ядра. Средой программирования послужил графический редактор Visual Studio Code~\cite{vscode}, так как в нём много плагинов, улучшающих процесс разработки.


\section{Реализация загружаемого модуля}

В листингах \ref{lst:md_init_1}--\ref{lst:md_init_2} представлена функция загрузки модуля, а в листинге \ref{lst:md_exit} функция выгрузки модуля.

\begin{lstlisting}[label=lst:md_init_1,caption=Функция загрузки модуля]
static int __init md_init(void)
{
	int err;
	
	err = alloc_lists();
	if (err)
	{
		return err;
	}
	
	err = init_proc();
	if (err)
	{
		free_proc();
		free_lists();
		return err;
		}
\end{lstlisting}
\begin{lstlisting}[label=lst:md_init_2,caption=Функция загрузки модуля]
	err = install_hooks();
	if(err)
	{
		printk(KERN_ERR "%s install_hooks error\n", PREFIX);
		free_proc();
		free_lists();
		
		return err;
	}
	
	pr_info("%s: module loaded!\n", PREFIX);
	
	return 0;
}

\end{lstlisting}


\begin{lstlisting}[label=lst:md_exit,caption=Функция выгрузки модуля]
static void __exit md_exit(void)
{
	remove_hooks();
	free_proc();
	free_lists();
	
	pr_info("%s: module unloaded!\n", PREFIX);
}
\end{lstlisting}

\clearpage

В листинге~\ref{lst:ftrace:lookup_name} представлена функция \texttt{lookup\_name()}, которая возвращающей адрес функции перехватываемой функции по её названию.

\begin{lstlisting}[label=lst:ftrace:lookup_name,caption=Реализация функции lookup\_name()]
#if LINUX_VERSION_CODE >= KERNEL_VERSION(5,7,0)
static unsigned long lookup_name(const char *name)
{
	struct kprobe kp = {
		.symbol_name = name
	};
	unsigned long retval;
	
	if (register_kprobe(&kp) < 0) return 0;
	retval = (unsigned long) kp.addr;
	unregister_kprobe(&kp);
	return retval;
}
#else
static unsigned long lookup_name(const char *name)
{
	return kallsyms_lookup_name(name);
}
#endif
\end{lstlisting}

В листингах~\ref{lst:ftrace:install_hook-1} и \ref{lst:ftrace:install_hook-2}  представлена реализация функции, которая инициализирует структуру \texttt{ftrace\_ops}.

\begin{lstlisting}[label=lst:ftrace:install_hook-1,caption=Реализация функции install\_hook()]
int fh_install_hook(struct ftrace_hook *hook)
{
	int err;
	
	err = fh_resolve_hook_address(hook);
	if (err)
		return err;
\end{lstlisting}	
\begin{lstlisting}[label=lst:ftrace:install_hook-2,caption=Реализация функции  fn\_install\_hook()]	
	hook->ops.func = fh_ftrace_thunk;
	hook->ops.flags = FTRACE_OPS_FL_SAVE_REGS
	| FTRACE_OPS_FL_RECURSION
	| FTRACE_OPS_FL_IPMODIFY;
	
	err = ftrace_set_filter_ip(&hook->ops, hook->address, 0, 0);
	if (err) {
		pr_debug("ftrace_set_filter_ip() failed: %d\n", err);
		return err;
	}
	
	err = register_ftrace_function(&hook->ops);
	if (err) {
		pr_debug("register_ftrace_function() failed: %d\n", err);
		ftrace_set_filter_ip(&hook->ops, hook->address, 1, 0);
		return err;
	}
	
	return 0;
}
\end{lstlisting}

В листингах \ref{lst:ftrace:remove_hook-1} и \ref{lst:ftrace:remove_hook-2} представлена реализация отключения перехвата функции.

\begin{lstlisting}[label=lst:ftrace:remove_hook-1,caption=Реализация функции fn\_remove\_hook()]
void fh_remove_hook(struct ftrace_hook *hook)
{
	int err;
	
	err = unregister_ftrace_function(&hook->ops);
	if (err) {
		pr_debug("unregister_ftrace_function() failed: %d\n", err);
	}
\end{lstlisting}
\begin{lstlisting}[label=lst:ftrace:remove_hook-2,caption=Реализация функции fn\_remove\_hook()]	
	err = ftrace_set_filter_ip(&hook->ops, hook->address, 1, 0);
	if (err) {
		pr_debug("ftrace_set_filter_ip() failed: %d\n", err);
	}
}
\end{lstlisting}

В листинге представлена реализация добавления перехватываемых функций в \texttt{hooks}.
\begin{lstlisting}[label=lst:ftrace:hooks,caption=Реализация добавления перехватываемых функций в \texttt{hooks}]	
static struct ftrace_hook hooks[] = {
	HOOK("sys_kill",  hook_sys_kill,  &real_sys_kill),
	HOOK("sys_signal",  hook_sys_signal,  &real_sys_signal),
	HOOK("sys_semget",  hook_sys_semget,  &real_sys_semget),
	HOOK("sys_semop",  hook_sys_semop,  &real_sys_semop),
	HOOK("sys_semctl",  hook_sys_semctl,  &real_sys_semctl),
	HOOK("sys_pipe",  hook_sys_pipe, &real_sys_pipe),
	HOOK("sys_pipe2",  hook_sys_pipe2, &real_sys_pipe2),
	HOOK("sys_close",  hook_sys_close, &real_sys_close),
	HOOK("sys_shmget",  hook_sys_shmget, &real_sys_shmget),
	HOOK("sys_shmat",  hook_sys_shmat, &real_sys_shmat),
	HOOK("sys_shmdt",  hook_sys_shmdt, &real_sys_shmdt),
	HOOK("sys_old_shmctl",  hook_sys_shmctl, &real_sys_shmctl),
	HOOK("sys_shmctl",  hook_sys_shmctl, &real_sys_shmctl),
};
\end{lstlisting}

В листингах~\ref{lst:sys_kill}--\ref{lst:sys_semctl-2} представлена реализация функций оберток \newline \texttt{sys\_kill}, \texttt{sys\_signal}, \texttt{sys\_semget}, \texttt{sys\_semop} и \texttt{sys\_semctl}.

\clearpage

\begin{lstlisting}[label=lst:sys_kill,caption=Реализация функции обертки \texttt{sys\_kill()}]	
static asmlinkage long (*real_sys_kill)(const struct pt_regs *);
static asmlinkage int hook_sys_kill(const struct pt_regs *regs)
{
	int res = real_sys_kill(regs);
	
	if (res == 0)
	{
		pid_t pid = regs->di;
		int sig = regs->si;
		
		char currentString[TEMP_STRING_SIZE];
		
		memset(currentString, 0, TEMP_STRING_SIZE);	
		snprintf(currentString, TEMP_STRING_SIZE, "Proccess %d sent signal %s to process %d\n", current->pid, signal_names[sig], pid);
		
		spin_lock(&signal_logs_lock);
		
		strcat(signal_logs, currentString); 
		
		spin_unlock(&signal_logs_lock);
	}
	
	return res;
}
\end{lstlisting}

\clearpage

\begin{lstlisting}[label=lst:sys_semget,caption=Реализация функции обертки \texttt{sys\_semget()}]	
static void get_sem_info(int semid, int nsems, int semflg)
{
	spin_lock(&sem_lock);
	for (int semnum = 0; semnum < nsems; semnum++) {
		sem_info_t info = {
			.semid = semid,
			.semnum = semnum + 1,
			.pid = current->pid,
			.semflg = semflg,
			.lastcmd = -1,
			.value = -1
		};
		
		push_bask_semlist(sem_info_list, info);
	}
	spin_unlock(&sem_lock);
}

static asmlinkage long (*real_sys_semget)(const struct pt_regs *);
static asmlinkage int hook_sys_semget(const struct pt_regs *regs)
{
	int semid = real_sys_semget(regs);
	
	key_t key = regs->di;
	int nsems = regs->si;
	int semflg = regs->dx;
	
	if (semid == -1) {
		pr_err("%s%s: Proccess %d can't create or get %d semafores\n", PREFIX, SEMPREFIX, current->pid, nsems);
	}
	else {
\end{lstlisting}		
\begin{lstlisting}[label=lst:sys_semget-2,caption=Реализация функции обертки \texttt{sys\_semget()}]
		get_sem_info(semid, nsems, semflg);
		pr_info("%s%s: Proccess %d create or get %d semafores with semid %d\n", PREFIX, SEMPREFIX, current->pid, nsems, semid);
	}
	
	return semid;
}
\end{lstlisting}


\begin{lstlisting}[label=lst:sys_semop,caption=Реализация функции обертки \texttt{sys\_semop()}]	
static void update_semop_info(int semid, struct sembuf __user *sops)
{
	spin_lock(&sem_lock);
	
	semnode *head = sem_info_list->head;
	
	for(;head; head = head->next) {
		if (head->info.semid == semid && head->info.semnum == sops->sem_num) {
			head->info.value += sops->sem_op;
		} 
	}
	
	spin_unlock(&sem_lock);
}

static asmlinkage long (*real_sys_semop)(const struct pt_regs *);
static asmlinkage int hook_sys_semop(const struct pt_regs *regs)
{
	int res = real_sys_semop(regs);
	
	int semid = regs->di;
\end{lstlisting}	
\begin{lstlisting}[label=lst:sys_semop-2,caption=Реализация функции обертки \texttt{sys\_semop()}]	
	struct sembuf __user *sops = regs->si;
	unsigned nsops = regs->dx;
		
	if (res == -1) {
		pr_err("%s%s: Proccess %d can't operate with %d semafore on semid %d\n", PREFIX, SEMPREFIX, current->pid, sops->sem_num, semid);
	}
	else {
		update_semop_info(semid, sops);
		pr_info("%s%s: Proccess %d operate with %d semafore on semid %d\n", PREFIX, SEMPREFIX, current->pid, sops->sem_num, semid);
	}
	
	return res;
}
\end{lstlisting}


\begin{lstlisting}[label=lst:sys_semctl,caption=Реализация функции обертки \texttt{sys\_semctl()}]	
static void update_semctl_info(int semid, int semnum, int cmd, unsigned long arg)
{
	spin_lock(&sem_lock);	
	semnode *head = sem_info_list->head;
	ushort *values = NULL;
	
	if (cmd == SETALL || cmd == GETALL)
		values = (ushort *) arg;

	for(;head; head = head->next) {
		if (head->info.semid == semid) {
			if (cmd == SETVAL && head->info.semnum == semnum + 1)
				head->info.value = arg;
\end{lstlisting}				
\begin{lstlisting}[label=lst:sys_semctl-2,caption=Реализация функции обертки \texttt{sys\_semctl()}]				
			else if (cmd == SETALL || cmd == GETALL)
				head->info.value = values[head->info.semnum - 1];	
			head->info.lastcmd = cmd;
		} 
	}
	spin_unlock(&sem_lock);
	
	pr_info("%s%s: Proccess %d semctl with %d semafore on semid %d, value: %d\n", PREFIX, SEMPREFIX, current->pid, semnum, semid, arg);
}

static asmlinkage long (*real_sys_semctl)(const struct pt_regs *);
static asmlinkage int hook_sys_semctl(const struct pt_regs *regs)
{
	int res = real_sys_semctl(regs);
	
	int semid = regs->di;
	int semnum = regs->si;
	int cmd = regs->dx;
	unsigned long arg = regs->r10;
	
	if (res == 0)
		update_semctl_info(semid, semnum, cmd, arg);
	
	return res;
}
\end{lstlisting}

\clearpage

В листингах~\ref{lst:read_general}--\ref{lst:read_general-2} представлена реализация чтения из файла general.

\begin{lstlisting}[label=lst:read_general, caption=Реализация функции чтения из файла general]			
static ssize_t general_read(struct file *file, char __user *buf, size_t len, loff_t *fPos)
{
	pr_info("%s%s: general_read called\n", PREFIX, FORTUNEPREFIX);
	if (*fPos > 0)
		return 0;
	
	ssize_t strlen += sprintf(general_info + strlen, "%7s %7s %7s %7s %10s %7s %7s %7s %7s %7s %14s %14s %14s %7s\n", 
	"PPID", "PID", "STATE", "ESTATE", "FLAGS", "POLICY", "PRIO", "SPRIO", "NPRIO", "PRPRIO", "UTIME", "STIME", "DELAY", "COMM");
	
	struct task_struct *task = &init_task;
	do {
		strlen += sprintf(general_info + strlen, "%7d %7d %7d %7d %10x %7d %7d %7d %7d %7d %14llu %14llu %14llu\t%s\n",
		task->parent->pid, task->pid, task->__state, task->exit_state, task->flags,
		task->policy, task->prio, task->static_prio, task->normal_prio, task->rt_priority,
		task->utime, task->stime, task->sched_info.run_delay,
		task->comm);  
	}
	while ((task = next_task(task)) != &init_task);
	
	if (copy_to_user(buf, general_info, strlen)) {
		printk(KERN_ERR "%s%s: copy_to_user error\n", PREFIX, FORTUNEPREFIX);	
		return -EFAULT;
	}
\end{lstlisting}	
\begin{lstlisting}[label=lst:read_general-2, caption=Реализация функции чтения из файла general]			
	*fPos += strlen;
	memset(general_info, 0, LOG_SIZE);
	
	return strlen;
}
\end{lstlisting}

В листингах~\ref{lst:read_sighands}--\ref{lst:read_sighands-2} представлена реализация чтения из файла sighands.

\begin{lstlisting}[label=lst:read_sighands, caption=Реализация функции чтения из файла sighands]			
static ssize_t sighand_read(struct file *file, char __user *buf, size_t len, loff_t *fPos)
{
	pr_info("%s%s: signal_read called\n", PREFIX, FORTUNEPREFIX);
	
	if (*fPos > 0)
		return 0;
	
	ssize_t strlen = 0;
	strlen += sprintf(sighand_info + strlen, "%7s\t%14s\t%8s\t%7s\t%7s\n", "PID", "SIGNAL", "FLAGS", "HANDLER");
	
	struct task_struct *task = &init_task;
	do {
		for (int signo = 1; signo < _NSIG; ++signo) {
			struct k_sigaction *ka = &task->sighand->action[signo - 1];
			
			if (ka->sa.sa_handler > 1) {
				strlen += sprintf(sighand_info + strlen, "%7d %14s %7lu 0x%x\n", task->pid, signal_names[signo], ka->sa.sa_flags, ka->sa.sa_handler);
\end{lstlisting}
\begin{lstlisting}[label=lst:read_sighand-2, caption=Реализация функции чтения из файла sighands]				
			}
		}
	}
	while ((task = next_task(task)) != &init_task);
	
	if (copy_to_user(buf, sighand_info, strlen)) {
		printk(KERN_ERR "%s: copy_to_user error\n", PREFIX);
		return -EFAULT;
	}
	
	*fPos += strlen;
	
	memset(sighand_info, 0, LOG_SIZE);
	
	return strlen;
}
\end{lstlisting}

В листингах~\ref{lst:read_memory}--\ref{lst:read_memory-2} представлена реализация чтения из файла memory.

\begin{lstlisting}[label=lst:read_memory, caption=Реализация функции чтения из файла memory]			
static ssize_t memory_read(struct file *file, char __user *buf, size_t len, loff_t *fPos)
{
	pr_info("%s%s: memory_read called\n", PREFIX, FORTUNEPREFIX);
	
	if (*fPos > 0)
		return 0;
	
	ssize_t strlen = 0;
	strlen += sprintf(memory_info + strlen, "%7s %7s %10s %10s %10s %10s %10s %7s %10s %10s %10s %10s %10s\n", 
	"PID", "MMUSERS", "TOTAL VM", "LOCKED VM", "DATA VM", "EXEC VM", "STACK VM", "MAPS", "HEAP", "CODE", "DATA", "ARGS", "ENV");
\end{lstlisting}	
\begin{lstlisting}[label=lst:read_memory-2, caption=Реализация функции чтения из файла memory]	
	struct task_struct *task = &init_task;
	do {
		struct mm_struct *mm = task->mm;
		
		if (mm != NULL) {
			unsigned long brk = mm->brk - mm->start_brk; 
			unsigned long code = mm->end_code - mm->start_code;   
			unsigned long data = mm->end_data - mm->start_data;
			unsigned long args = mm->arg_end - mm->arg_start;
			unsigned long env = mm->env_end - mm->env_start;
			
			strlen += sprintf(memory_info + strlen, "%7d %7d %10lu %10lu %10lu %10lu %10lu %7d %10lu %10lu %10lu %10lu %10lu\n", 
			task->pid, mm->mm_users.counter, mm->total_vm, mm->locked_vm, mm->data_vm, mm->exec_vm, mm->stack_vm, mm->map_count, brk, code, data, args, env);
		}
	}
	while ((task = next_task(task)) != &init_task);
	
	if (copy_to_user(buf, memory_info, strlen)){
		printk(KERN_ERR "%s: copy_to_user error\n", PREFIX);
		return -EFAULT;
	}
	
	*fPos += strlen;
	memset(memory_info, 0, LOG_SIZE);
	
	return strlen;
}
\end{lstlisting}

В листинге~\ref{lst:write_maps} представлена реализация записи в файл maps.

\begin{lstlisting}[label=lst:write_maps, caption=Реализация функции записи в файл maps]			
static ssize_t maps_write(struct file *file, const char __user *ubuf, size_t len, loff_t *fPos)
{
	pr_info("%s%s: maps_write called\n", PREFIX, FORTUNEPREFIX);
	
	char kbuf[10];
	if (copy_from_user(kbuf, ubuf, len)){
		printk(KERN_ERR "%s%s: copy_from_user error\n", PREFIX, FORTUNEPREFIX);
		return -EFAULT;
	}
	kbuf[len - 1] = 0;
	
	if(sscanf(kbuf, "%d", &mt_pid) != 1)
	{
		printk(KERN_ERR "%s: sscanf error\n", PREFIX);
		return -EFAULT;
	}
	
	return len;
}
\end{lstlisting}

В листингах~\ref{lst:read_maps}--\ref{lst:read_maps-3} представлена реализация чтения из файла maps.

\begin{lstlisting}[label=lst:read_maps, caption=Реализация функции чтения из файла maps]			
static ssize_t maps_read(struct file *file, char __user *buf, size_t len, loff_t *fPos)
{
	pr_info("%s%s: maps_read called\n", PREFIX, FORTUNEPREFIX);
	
	if (*fPos > 0)
		return 0;
	
	ssize_t strlen = 0;
\end{lstlisting}	
\begin{lstlisting}[label=lst:read_maps-2, caption=Реализация функции чтения из файла maps]		
	struct task_struct *task = find_task_struct(mt_pid);
	
	if (task == NULL || mt_pid == -1)
	{
		strlen += sprintf(maps_info + strlen, "Process with pid %d doesn't exist\n", mt_pid);
	}
	else
	{
		strlen += sprintf(maps_info + strlen, "%7s %15s %10s %10s %10s\n", "PID", "addr-addr", "Flags", "BYTES", "PAGES");
		
		struct mm_struct *mm = task->mm;
		
		if (mm == NULL)
			strlen += sprintf(maps_info + strlen, "%7d %20s %10s %10s %10s\n", task->pid, "?-?", "?", "?", "?");
		else{
			struct vm_area_struct *vma = mm->mmap;
			
			if (vma == NULL)
				strlen += sprintf(maps_info + strlen, "%7d %15s %10s %10s %10s\n", task->pid, "?-?", "?", "?", "?");
			else {
				for (; vma != NULL; vma = vma->vm_next){
					unsigned long bytes = vma->vm_end - vma->vm_start;
					int pages = bytes / 4096;
					
					strlen += sprintf(maps_info + strlen, "%7d %x-%x %10lld %10lu %7d\n", 
\end{lstlisting}	
\begin{lstlisting}[label=lst:read_maps-3, caption=Реализация функции чтения из файла maps]					
					task->pid, vma->vm_start, vma->vm_end, vma->vm_flags, bytes, pages);
				}
			}
		}
	}
	
	if (copy_to_user(buf, maps_info, strlen)) {
		printk(KERN_ERR "%s%s: copy_to_user error\n", PREFIX, FORTUNEPREFIX);
		return -EFAULT;
	}
	
	memset(maps_info, 0, LOG_SIZE);
	
	*fPos += strlen;
	
	return strlen;
}
\end{lstlisting}

В листингах~\ref{lst:read_pipe}--\ref{lst:read_pipe-2} представлена реализация чтения из файла pipe.

\begin{lstlisting}[label=lst:read_pipe, caption=Реализация функции чтения из файла pipe]			
static ssize_t pipe_read(struct file *file, char __user *buf, size_t len, loff_t *fPos)
{
	pr_info("%s%s: pipe_read called\n", PREFIX, FORTUNEPREFIX);
	
	if (*fPos > 0)
		return 0;
	
	ssize_t strlen = 0;
	strlen += sprintf(pipes_info + strlen, "%7s %18s %7s %s\n", "PID", "FD", "COUNT", "PIDS");
\end{lstlisting}	
\begin{lstlisting}[label=lst:read_pipe-2, caption=Реализация функции чтения из файла pipe]		
	pnode *head = pipe_info_list.head;
	
	for (; head; head = head->next) {
		int count = 0;
		ssize_t clen = 0;
		char temp[TEMP_STRING_SIZE] = { 0 };
		
		list_head *pos;
		task_struct *task, *child;
		task = pid_task(find_vpid(head->pid), PIDTYPE_PID);
		
		list_for_each(pos; task->children; list) {
			child = list_entry(pos; struct task_struct, sibling);
			clen += sprintf(temp + clen, "%d,", child->pid);
			count++;
		}
		
		strlen += sprintf(pipes_info + strlen, "%7d %18llu %7d %s\n",
		head->ppid, head->fd, count, temp);   
	}
	
	if (copy_to_user(buf, pipes_info, strlen)) {
		printk(KERN_ERR "%s%s: copy_to_user error\n", PREFIX, FORTUNEPREFIX);
		return -EFAULT;
	}
		
	*fPos += strlen;
	memset(pipes_info, 0, LOG_SIZE);
	
	return strlen;
}
\end{lstlisting}

В листинге~\ref{lst:read_sem} представлена реализация чтения из файла sem.

\begin{lstlisting}[label=lst:read_sem, caption=Реализация функции чтения из файла sem]			
static ssize_t sem_read(struct file *file, char __user *buf, size_t len, loff_t *fPos)
{
	pr_info("%s%s: sem_read called\n", PREFIX, FORTUNEPREFIX);
	
	if (*fPos > 0)
		return 0;
	
	ssize_t strlen = 0;
	strlen += sprintf(sem_info + strlen, "%7s %7s %7s %7s %7s %7s\n", "PID", "SEMID", "SEMNUM", "FLAGS", "CMD", "VALUE");
	semnode *head = sem_info_list->head;
	for (; head; head = head->next) {
		char command[10] = { 0 };    
		cmd_to_str(command, head->info.lastcmd);
		
		strlen += sprintf(sem_info + strlen, "%7d %7d %7d %7d %7s %7d\n",
		head->info.pid, head->info.semid, head->info.semnum, head->info.semflg, command, head->info.value);    
	}
	
	if (copy_to_user(buf, sem_info, strlen)) {
		printk(KERN_ERR "%s%s: copy_to_user error\n", PREFIX, FORTUNEPREFIX);
		return -EFAULT;
	}
		
	*fPos += strlen;
	memset(sem_info, 0, LOG_SIZE);
	return strlen;
}
\end{lstlisting}

В листингах~\ref{lst:read_shm}--\ref{lst:read_shm-2} представлена реализация чтения из файла pipe.

\begin{lstlisting}[label=lst:read_shm, caption=Реализация функции чтения из файла shm]			
static ssize_t shm_read(struct file *file, char __user *buf, size_t len, loff_t *fPos)
{
	pr_info("%s%s: shm_read called\n", PREFIX, FORTUNEPREFIX);
	
	if (*fPos > 0)
		return 0;
	
	ssize_t strlen = 0;
	strlen += sprintf(shm_info + strlen, "%7s %7s %10s %14s %7s\n", "PID", "SHMID", "CMD", "SIZE", "ADDR");
	shmnode *head = shm_info_list->head;
	for (; head; head = head->next)	{
		char command[10] = { 0 };    
		cmd_to_str(command, head->info.lastcmd);
		
		if (head->info.addr == NULL) {
			strlen += sprintf(shm_info + strlen, "%7d %7d %10s %14llu %s\n",
			head->info.pid, head->info.shmid, command, head->info.size, "?"); 
		}
		else { 
			strlen += sprintf(shm_info + strlen, "%7d %7d %10s %14llu 0x%p\n",
			head->info.pid, head->info.shmid, command, head->info.size, head->info.addr);
		}
	}
	if (copy_to_user(buf, shm_info, strlen)) {
		printk(KERN_ERR "%s%s: copy_to_user error\n", PREFIX, FORTUNEPREFIX);
\end{lstlisting}
\begin{lstlisting}[label=lst:read_shm-2, caption=Реализация функции чтения из файла shm]			
		return -EFAULT;
	}
	
	*fPos += strlen;
	memset(shm_info, 0, LOG_SIZE);
	return strlen;
}
\end{lstlisting}

Для файлов были созданы экземпляры структуры \texttt{proc\_ops}, они представлены в листингах~\ref{lst:proc_ops}--\ref{lst:proc_ops-2}.

\begin{lstlisting}[label=lst:proc_ops, caption=Экземпляры структуры proc\_ops]			
static struct proc_ops signal_logs_ops = {
	.proc_open = signal_logs_open,
	.proc_read = signal_logs_read,
	.proc_write = signal_logs_write,
	.proc_release = signal_logs_release,
};
static struct proc_ops sighand_ops = {
	.proc_open = sighand_open,
	.proc_read = sighand_read,
	.proc_write = sighand_write,
	.proc_release = sighand_release,
};
static struct proc_ops memory_ops = {
	.proc_open = memory_open,
	.proc_read = memory_read,
	.proc_write = memory_write,
	.proc_release = memory_release,
};
static struct proc_ops maps_ops = {
	.proc_open = maps_open,
	.proc_read = maps_read,
	.proc_write = maps_write,
\end{lstlisting}	
\begin{lstlisting}[label=lst:proc_ops-2, caption=Экземпляры структуры proc\_ops]	
	.proc_release = maps_release,
};
static struct proc_ops general_ops = {
	.proc_open = general_open,
	.proc_read = general_read,
	.proc_write = general_write,
	.proc_release = general_release,
};
static struct proc_ops pipe_ops = {
	.proc_open = pipe_open,
	.proc_read = pipe_read,
	.proc_write = pipe_write,
	.proc_release = pipe_release,
};
static struct proc_ops sem_ops = {
	.proc_open = sem_open,
	.proc_read = sem_read,
	.proc_write = sem_write,
	.proc_release = sem_release,
};
static struct proc_ops shm_ops = {
	.proc_open = shm_open,
	.proc_read = shm_read,
	.proc_write = shm_write,
	.proc_release = shm_release,
};
\end{lstlisting}

Весь код программы представлен в Приложении А.